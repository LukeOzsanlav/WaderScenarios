% Options for packages loaded elsewhere
\PassOptionsToPackage{unicode}{hyperref}
\PassOptionsToPackage{hyphens}{url}
\PassOptionsToPackage{dvipsnames,svgnames,x11names}{xcolor}
%
\documentclass[
  12pt,
  letterpaper,
  DIV=11,
  numbers=noendperiod]{scrartcl}

\usepackage{amsmath,amssymb}
\usepackage{iftex}
\ifPDFTeX
  \usepackage[T1]{fontenc}
  \usepackage[utf8]{inputenc}
  \usepackage{textcomp} % provide euro and other symbols
\else % if luatex or xetex
  \usepackage{unicode-math}
  \defaultfontfeatures{Scale=MatchLowercase}
  \defaultfontfeatures[\rmfamily]{Ligatures=TeX,Scale=1}
\fi
\usepackage{lmodern}
\ifPDFTeX\else  
    % xetex/luatex font selection
\fi
% Use upquote if available, for straight quotes in verbatim environments
\IfFileExists{upquote.sty}{\usepackage{upquote}}{}
\IfFileExists{microtype.sty}{% use microtype if available
  \usepackage[]{microtype}
  \UseMicrotypeSet[protrusion]{basicmath} % disable protrusion for tt fonts
}{}
\makeatletter
\@ifundefined{KOMAClassName}{% if non-KOMA class
  \IfFileExists{parskip.sty}{%
    \usepackage{parskip}
  }{% else
    \setlength{\parindent}{0pt}
    \setlength{\parskip}{6pt plus 2pt minus 1pt}}
}{% if KOMA class
  \KOMAoptions{parskip=half}}
\makeatother
\usepackage{xcolor}
\usepackage[top=30mm,left=30mm]{geometry}
\setlength{\emergencystretch}{3em} % prevent overfull lines
\setcounter{secnumdepth}{-\maxdimen} % remove section numbering
% Make \paragraph and \subparagraph free-standing
\makeatletter
\ifx\paragraph\undefined\else
  \let\oldparagraph\paragraph
  \renewcommand{\paragraph}{
    \@ifstar
      \xxxParagraphStar
      \xxxParagraphNoStar
  }
  \newcommand{\xxxParagraphStar}[1]{\oldparagraph*{#1}\mbox{}}
  \newcommand{\xxxParagraphNoStar}[1]{\oldparagraph{#1}\mbox{}}
\fi
\ifx\subparagraph\undefined\else
  \let\oldsubparagraph\subparagraph
  \renewcommand{\subparagraph}{
    \@ifstar
      \xxxSubParagraphStar
      \xxxSubParagraphNoStar
  }
  \newcommand{\xxxSubParagraphStar}[1]{\oldsubparagraph*{#1}\mbox{}}
  \newcommand{\xxxSubParagraphNoStar}[1]{\oldsubparagraph{#1}\mbox{}}
\fi
\makeatother


\providecommand{\tightlist}{%
  \setlength{\itemsep}{0pt}\setlength{\parskip}{0pt}}\usepackage{longtable,booktabs,array}
\usepackage{calc} % for calculating minipage widths
% Correct order of tables after \paragraph or \subparagraph
\usepackage{etoolbox}
\makeatletter
\patchcmd\longtable{\par}{\if@noskipsec\mbox{}\fi\par}{}{}
\makeatother
% Allow footnotes in longtable head/foot
\IfFileExists{footnotehyper.sty}{\usepackage{footnotehyper}}{\usepackage{footnote}}
\makesavenoteenv{longtable}
\usepackage{graphicx}
\makeatletter
\def\maxwidth{\ifdim\Gin@nat@width>\linewidth\linewidth\else\Gin@nat@width\fi}
\def\maxheight{\ifdim\Gin@nat@height>\textheight\textheight\else\Gin@nat@height\fi}
\makeatother
% Scale images if necessary, so that they will not overflow the page
% margins by default, and it is still possible to overwrite the defaults
% using explicit options in \includegraphics[width, height, ...]{}
\setkeys{Gin}{width=\maxwidth,height=\maxheight,keepaspectratio}
% Set default figure placement to htbp
\makeatletter
\def\fps@figure{htbp}
\makeatother
% definitions for citeproc citations
\NewDocumentCommand\citeproctext{}{}
\NewDocumentCommand\citeproc{mm}{%
  \begingroup\def\citeproctext{#2}\cite{#1}\endgroup}
\makeatletter
 % allow citations to break across lines
 \let\@cite@ofmt\@firstofone
 % avoid brackets around text for \cite:
 \def\@biblabel#1{}
 \def\@cite#1#2{{#1\if@tempswa , #2\fi}}
\makeatother
\newlength{\cslhangindent}
\setlength{\cslhangindent}{1.5em}
\newlength{\csllabelwidth}
\setlength{\csllabelwidth}{3em}
\newenvironment{CSLReferences}[2] % #1 hanging-indent, #2 entry-spacing
 {\begin{list}{}{%
  \setlength{\itemindent}{0pt}
  \setlength{\leftmargin}{0pt}
  \setlength{\parsep}{0pt}
  % turn on hanging indent if param 1 is 1
  \ifodd #1
   \setlength{\leftmargin}{\cslhangindent}
   \setlength{\itemindent}{-1\cslhangindent}
  \fi
  % set entry spacing
  \setlength{\itemsep}{#2\baselineskip}}}
 {\end{list}}
\usepackage{calc}
\newcommand{\CSLBlock}[1]{\hfill\break\parbox[t]{\linewidth}{\strut\ignorespaces#1\strut}}
\newcommand{\CSLLeftMargin}[1]{\parbox[t]{\csllabelwidth}{\strut#1\strut}}
\newcommand{\CSLRightInline}[1]{\parbox[t]{\linewidth - \csllabelwidth}{\strut#1\strut}}
\newcommand{\CSLIndent}[1]{\hspace{\cslhangindent}#1}

\usepackage{booktabs}
\usepackage{longtable}
\usepackage{array}
\usepackage{multirow}
\usepackage{wrapfig}
\usepackage{float}
\usepackage{colortbl}
\usepackage{pdflscape}
\usepackage{tabu}
\usepackage{threeparttable}
\usepackage{threeparttablex}
\usepackage[normalem]{ulem}
\usepackage{makecell}
\usepackage{xcolor}
\usepackage{scrlayer-scrpage}
\rohead{\includegraphics[height=1.4cm]{qmdimages/RSPB_logo.png}}
\lofoot{North Kent}
\KOMAoption{captions}{tableheading}
\usepackage{float} \floatplacement{figure}{H} \newcommand{\beginsupplement}{\setcounter{table}{0}  \renewcommand{\thetable}{S\arabic{table}} \setcounter{figure}{0} \renewcommand{\thefigure}{S\arabic{figure}}}
\makeatletter
\@ifpackageloaded{caption}{}{\usepackage{caption}}
\AtBeginDocument{%
\ifdefined\contentsname
  \renewcommand*\contentsname{Table of contents}
\else
  \newcommand\contentsname{Table of contents}
\fi
\ifdefined\listfigurename
  \renewcommand*\listfigurename{List of Figures}
\else
  \newcommand\listfigurename{List of Figures}
\fi
\ifdefined\listtablename
  \renewcommand*\listtablename{List of Tables}
\else
  \newcommand\listtablename{List of Tables}
\fi
\ifdefined\figurename
  \renewcommand*\figurename{Figure}
\else
  \newcommand\figurename{Figure}
\fi
\ifdefined\tablename
  \renewcommand*\tablename{Table}
\else
  \newcommand\tablename{Table}
\fi
}
\@ifpackageloaded{float}{}{\usepackage{float}}
\floatstyle{ruled}
\@ifundefined{c@chapter}{\newfloat{codelisting}{h}{lop}}{\newfloat{codelisting}{h}{lop}[chapter]}
\floatname{codelisting}{Listing}
\newcommand*\listoflistings{\listof{codelisting}{List of Listings}}
\makeatother
\makeatletter
\makeatother
\makeatletter
\@ifpackageloaded{caption}{}{\usepackage{caption}}
\@ifpackageloaded{subcaption}{}{\usepackage{subcaption}}
\makeatother

\ifLuaTeX
  \usepackage{selnolig}  % disable illegal ligatures
\fi
\usepackage{bookmark}

\IfFileExists{xurl.sty}{\usepackage{xurl}}{} % add URL line breaks if available
\urlstyle{same} % disable monospaced font for URLs
\hypersetup{
  pdftitle={Stakeholder preferences for breeding wader conservation across North Kent},
  pdfauthor={Luke Ozsanlav-Harris \& Robert Hawkes},
  colorlinks=true,
  linkcolor={blue},
  filecolor={Maroon},
  citecolor={Blue},
  urlcolor={Blue},
  pdfcreator={LaTeX via pandoc}}


\title{Stakeholder preferences for breeding wader conservation across
North Kent}
\author{Luke Ozsanlav-Harris \& Robert Hawkes}
\date{2024-11-08}

\begin{document}
\maketitle


\section{Summary}\label{summary}

This report presents the outputs from stakeholder workshops that aimed
to explore where lowland wader conservation would preferably be carried
out in North Kent. We held a workshop with a varied group of local
stakeholders that we were broadly divided into three groups:
conservationists, public bodies and land managers. During the workshops
each stakeholder group created a series of guidelines that could be used
to identify areas within the landscape where wader conservation would be
preferentially targeted or avoided. As part of the workshops we also
introduced participants to three key components of the ``Lawton
principles of nature restoration'' (\citeproc{ref-lawton2010}{Lawton et
al. 2010}), i.e.~improving the quality of current wildlife sites by
better management (`better'), increasing the size of current wildlife
sites (`bigger'), and creating new wildlife sites (`more'). This was to
allow stakeholders to make their guidelines specific to different
principles, thereby trying to transition from theory to practical
conservation. Using the guidelines generated in each workshop we
generated spatial graded layers that were then combined to rank land in
order of preference for each landscape and (where possible) each
stakeholder group. These heat maps indicate where wader conservation
would be most preferred within each landscape under each of the three
Lawton principles.

\textbf{Correspondence:} robert.hawkes@rspb.org.uk

\newpage{}

\section{Introduction}\label{introduction}

Many areas across England have been heavily modified by humans and
contain a mixture of land uses and semi-natural habitats
(\citeproc{ref-song2018}{Song et al. 2018}). Modified landscapes often
harbor important biodiversity, but ongoing intensification of land
management threatens what biodiversity remains
(\citeproc{ref-raven2021}{Raven and Wagner 2021};
\citeproc{ref-donald2001}{Donald, Green, and Heath 2001}). Due to a
variety of different land uses in modified landscapes, restoration of
biodiversity must compete with numerous other objectives such as, food
and timber production, energy production, water resource management and
urbanization (\citeproc{ref-chazdon2016}{Chazdon et al. 2016}).
Restoring biodiversity at larger scales must therefore carefully balance
multiple objectives and working alongside the people living and working
in the landscape is an integral part of this process. Involving local
stakeholders from the start of ecological restoration and adapting the
process along the way with ongoing engagement can help lead to higher
levels of mutual understanding as well enhancing mutual benefits
(\citeproc{ref-gamborg2019}{Gamborg, Morsing, and Raulund-Rasmussen
2019}).

Lowland breeding waders across Europe (order \emph{Charadriiformes})
declined during the 20th century
(\citeproc{ref-roodbergen2011}{Roodbergen, Werf, and Hötker 2011}) due
to loss and degradation of their preferred habitat, floodplain and
coastal grasslands. Habitat was lost due land use change and remaining
areas became degraded due to drainage (reduced prey availability,
(\citeproc{ref-eglington2010}{Eglington et al. 2010})); increased
stocking and earlier more frequent mowing (nest and brood destruction,
(\citeproc{ref-sabatier2010}{Sabatier, Doyen, and Tichit 2010})); and
increased meso-predator density (eggs and chick predation,
(\citeproc{ref-roos2018}{Roos et al. 2018})). Across much of Europe
efforts to restore populations have taken two main avenues:

\begin{enumerate}
\def\labelenumi{\arabic{enumi}.}
\item
  creation of Protected Areas (PA) through statutory protection
  (e.g.~Sites of Special Scientific Interest, SSSI) and nature reserves.
  On nature reserves specifically, high-quality breeding habitat is
  often created through rewetting
  (\citeproc{ref-eglington2010}{Eglington et al. 2010}), appropriate
  grazing management (\citeproc{ref-verhulst2011}{Verhulst et al. 2011})
  and predator exclusion with fences (\citeproc{ref-malpas2013}{Malpas
  et al. 2013}). Statutory protection (specifically SSSIs) on its own
  provides weaker benefits to breeding waders
  (\citeproc{ref-smart2014}{Smart et al. 2014};
  \citeproc{ref-hawkes2024}{Hawes 2024}).
\item
  use of agri-environment schemes on farmed land where payments support
  bespoke wader-focused measures like that of reserves, but generally
  lacking predator fences. This produces lesser quality habitat
  (\citeproc{ref-smart2014}{Smart et al. 2014}) but it is cheaper to
  create and maintain and often fits in with the existing land user
  operations (e.g.~beef cattle farming and hay/silage cutting).
\end{enumerate}

Despite widespread uptake of AES and reserve/PA creation, populations of
lowland breeding waders have still declined across Europe
(\citeproc{ref-franks2018}{Franks et al. 2018}). Suggesting improvements
in efficacy or further increases in scale are needed to prevent further
population declines. The `Making Space for Nature' report
(\citeproc{ref-lawton2010}{Lawton et al. 2010}) set out a spatial
targeting approach for landscape-scale restoration that was distilled
down to four words, `better, bigger, more and joined'. Based off the
summation of a substantial body of scientific work this report
recommended these actions in order of priority: (1) improving the
quality of existing habitat, (2) increasing the size and (3) number of
sites, and (4) enhancing connectivity among sites for conservation. This
provides a starting framework but there are important trade-offs between
strategies imposed by limited resources, land, and surrounding context
(e.g.~wider land-uses, levels of fragmentation and biogeographic
context). We focused on the first three principles during workshops as
wading birds can likely readily disperse between the existing habitat
patches within our landscapes of interest
(\citeproc{ref-jackson1994}{Jackson 1994}).

\begin{figure}[H]

\centering{

\includegraphics[width=4.6875in,height=\textheight]{qmdimages/AllLandscapePlots.png}

}

\caption{\label{fig-landscapes}Map of the four case-study landscapes
from this study. Essex and North Kent were formerly part of the same
``priority landscape'' but split due to differing characteristics and
geographic seperation.}

\end{figure}%

We ran stakeholder workshops across four different landscapes to
understand where it is possible to deploy these strategies according to
people who live and work in the area, thereby trying to link theory to
real world situations and landscapes. We focus on three different
`priority landscapes' (formerly known as Environmentally Sensitive
Areas, (\citeproc{ref-nateng2024}{Natural England 2024})): Somerset
Levels; Norfolk Broads (hereafter Broads); and Greater Thames (see
Figure~\ref{fig-landscapes}). We subsequently split the Greater Thames
landscape up into North Kent and Essex owing to differing land uses and
geographic separation. We chose these landscapes as they hold important
populations of breeding waders in a national context
(\citeproc{ref-wilson2005}{Wilson et al. 2005}) but also have different
characteristics (e.g.~soil types, land use compositions, wading bird
assemblages, uptake of AES and lowland wet grassland distribution, see
Table~\ref{tbl-scapestats} for landscape specific details). For North
Kent, Essex and the Broads we focus on Lapwing \emph{Vanellus vanellus}
and Redshank \emph{Tringa totanus} during the workshops and for the
Somerset Levels we focused on Lapwing and Snipe \emph{Gallinago
gallinago}. We focused on these species as they were the predominant
breeding wader species in each landscape. Curlews are a key part of the
breeding wader assemblage in the Somerset Levels but there were too few
survey records to build a model that predicted abundance, which was a
key aspect of the wider project.

Extensive breeding wader surveys were conducted in each landscape in
2021-22 as part of the national breeding wader of wet meadows survey. Up
to three visits during the breeding season were made to lowland wet
grassland fields (areas below 200m altitude subject to freshwater
flooding and water logging) with wading bird abundance and habitat
characteristics recorded. An optional fourth `dusk' visit was also
undertaken where Snipe breeding was suspected. As part of the wider
project, the four selected landscapes also received further visits in
2023 to survey areas that were missed in 2021-22.

The remainder of the reports details the methods of how the workshop
were run and how we turned stakeholder preferences into spatial maps for
all landscapes. We then present the heat map of stakeholder preferences
for just North Kent.

\begin{table}

\caption{\label{tbl-scapestats}Charactericists for each of the four case
study landscapes. The breeding pairs for each landscapes were estimated
from the breeding waders of wet meadows survey. The total hectarage for
AES excludes reserves that also have AES agreements.}

\centering{

\centering\begingroup\fontsize{10}{12}\selectfont

\begin{tabular}[t]{>{\raggedright\arraybackslash}p{11.5em}>{\raggedright\arraybackslash}p{7.25em}>{\raggedright\arraybackslash}p{7.25em}>{\raggedright\arraybackslash}p{7.25em}>{\raggedright\arraybackslash}p{7.25em}}
\toprule
\begingroup\fontsize{12}{14}\selectfont \textbf{ }\endgroup & \begingroup\fontsize{12}{14}\selectfont \textbf{Broads}\endgroup & \begingroup\fontsize{12}{14}\selectfont \textbf{Essex}\endgroup & \begingroup\fontsize{12}{14}\selectfont \textbf{North Kent}\endgroup & \begingroup\fontsize{12}{14}\selectfont \textbf{Somerset}\endgroup\\
\midrule
\cellcolor{gray!10}{Total Area (ha)} & \cellcolor{gray!10}{43,138} & \cellcolor{gray!10}{72,342} & \cellcolor{gray!10}{22,798} & \cellcolor{gray!10}{30,905}\\
\addlinespace
Reserve LWG area (ha) & 1,151 & 1,636 & 2,151 & 1,204\\
\addlinespace
\cellcolor{gray!10}{AES only LWG area (ha)} & \cellcolor{gray!10}{5,068} & \cellcolor{gray!10}{1,279} & \cellcolor{gray!10}{1,976 ha} & \cellcolor{gray!10}{3,078}\\
\addlinespace
LWG type & Floodplain & Coastal & Coastal & Floodplain\\
\addlinespace
\cellcolor{gray!10}{Land use} & \cellcolor{gray!10}{Mixed arable \& grassland} & \cellcolor{gray!10}{Mainly arable, some grassland} & \cellcolor{gray!10}{Mainly grassland, some arable} & \cellcolor{gray!10}{Mainly grassland}\\
\addlinespace
Soil Type & Mixed mineral \& organic & Mainly mineral & Mainly mineral & Mainly organic, some mineral\\
\addlinespace
\cellcolor{gray!10}{Breeding Pairs} & \cellcolor{gray!10}{775} & \cellcolor{gray!10}{825} & \cellcolor{gray!10}{1575} & \cellcolor{gray!10}{225}\\
\addlinespace
Predominant Species & Lapwing/ Redshank & Lapwing/ Redshank & Lapwing/ Redshank & Snipe\\
\bottomrule
\end{tabular}
\endgroup{}

}

\end{table}%

\section{Methods}\label{methods}

\subsection{Workshop Aim}\label{workshop-aim}

To create a map of future opportunity for wader conservation for each
stakeholder group within each landscape.

This can consist of preferences of where wader conservation could occur
as well as defining areas where wader conservation should be avoided.
These preferences can also be linked to specific nature restoration
strategies (better, bigger, more) and where arable land could be
reverted back to lowland wet grassland so that the map of future
opportunity can vary depending on the strategy.

\subsection{Workshop Attendees}\label{workshop-attendees}

We ran four regional workshops and we aimed to have three different
stakeholder groups attend each workshop. The three different groups (and
organisations invited) were:

\begin{itemize}
\tightlist
\item
  Conservationists (RSPB, Wildlife Trust, Wildfowl and wetland trust and
  Private nature reserves)
\item
  Public bodies (Natural England, Environment Agency, Internal drainage
  board, Local Authorities, Ministry of Defense, regional Farming and
  Wildlife Advisory Group rep)
\item
  Land managers (land owners, farmers, tenant farmers)
\end{itemize}

These groups were created to align participants in terms of background.
This helped to drive more productive group discussions and made it more
likely that a consensus would be reached during group activities. The
people invited to the workshops were largely already known to regional
RSPB members of staff. This may slightly bias the group of people that
attended towards those with more of an understanding and preference for
conservation. This could have particularly affected the land managers
group, for example some tenant farmers rent RSPB land for cattle grazing
or carry out wader friendly management on their own farm. This could
result in outputs are not fully representative of the wider stakeholder
group. Overall, we felt that this bias was tolerable and that a group of
more like minded participants would lead to more productive
conversations and ultimately result in more usable outputs.

In the end it was not possible to run all three group of stakeholders in
each priority landscape, apart from the Somerset Levels. We were only
able to run the activities with two groups in three of the landscape,
see Table~\ref{tbl-GrpAtt} for a breakdown of group attendance.

\begin{table}

\caption{\label{tbl-GrpAtt}Workshop attendance for the three different
stakeholder groups across the four priority landscapes. Note if only one
member of a stakeholder group attended a workshop then this individual
was gnerally moved into one of the other groups. Number of attendees for
each group is shown in brackets.}

\centering{

\centering\begingroup\fontsize{10}{12}\selectfont

\begin{tabular}[t]{llll}
\toprule
\begingroup\fontsize{12}{14}\selectfont \textbf{Landscape}\endgroup & \begingroup\fontsize{12}{14}\selectfont \textbf{Conservationists}\endgroup & \begingroup\fontsize{12}{14}\selectfont \textbf{Public Bodies}\endgroup & \begingroup\fontsize{12}{14}\selectfont \textbf{Land Managers}\endgroup\\
\midrule
\cellcolor{gray!10}{Broads} & \cellcolor{gray!10}{Y (5)} & \cellcolor{gray!10}{Y (7)} & \cellcolor{gray!10}{}\\
\addlinespace
Kent & Y (5) &  & Y (7)\\
\addlinespace
\cellcolor{gray!10}{Essex} & \cellcolor{gray!10}{Y (7)} & \cellcolor{gray!10}{} & \cellcolor{gray!10}{Y (7)}\\
\addlinespace
Somerset & Y (7) & Y (6) & Y (8)\\
\bottomrule
\end{tabular}
\endgroup{}

}

\end{table}%

\subsection{Workshop Activities}\label{workshop-activities}

In each workshop we gave an introductory presentation followed by three
stakeholder-led activities. The introductory presentation was in two
parts. Initially we presented the results from the 2021/22 breeding
waders of wet meadow survey, including the influence of habitat and land
management on breeding populations and the distribution of populations
within the landscape. Maps were also provided to participants to the
show the distribution of breeding wader populations and the layout of
different land uses within the priority landscape. Throughout all
activities, we told participants to focus on land within the priority
landscape; that the conservation of Lapwing and Redshank was a priority
(Lapwing and Snipe in the Somerset Levels); and to imagine what could be
possible in the year 2050.

After the presentation we ran three activities. We show below how each
task was presented to participants during the workshops.

\begin{itemize}
\tightlist
\item
  Activity 1

  \begin{itemize}
  \tightlist
  \item
    For each wader habitat intervention card discuss the challenges and
    opportunities. These can be associated with certain areas,
    land-uses, farming practices, costs/funding or practicalities.
    Mediators will record your discussion on the back of each card.
  \item
    After the cards provided record any other interventions on the blank
    cards provided and discuss their challenges/opportunities.
  \item
    The cards depicted the following interventions or management
    strategies for breeding waders: keeping standing water throughout
    spring; foot drain/scrape creation for wet surface features; grazing
    to create a varied sward; rush control; delayed cutting on
    hay/silage fields; predator exclusion using fences; and predator
    control.
  \end{itemize}
\item
  Activity 2

  \begin{itemize}
  \tightlist
  \item
    Discuss any goals for breeding waders. For example, how many waders,
    in the landscape and which species?
  \item
    Are their existing landscape plans that could influence breeding
    waders?
  \item
    Choose conservation strategies and rank them (better, bigger, more
    and arable reversion). If you can't rank strategies choose priority
    ones.
  \end{itemize}
\item
  Activity 3

  \begin{itemize}
  \tightlist
  \item
    Create guidelines for where each strategy can (preference) and can't
    (avoidance) be used.
  \item
    Mention any data sources that could be used to create the
    guidelines. Are there specific cut-off points associated with any of
    the guidelines?
  \end{itemize}
\end{itemize}

Activity 1 was designed as a primer activity and while we recorded the
main points of discussions within stakeholder groups there were no main
outputs from this activity. This activity was designed to initiate
conversations about wader conservation and the challenges and
opportunities in its implementation. This activity helped spark ideas
for further activities as it identified where management for breeding
waders would be the easiest or hardest to implement (see activity 3).

Activity 2 was designed as another primer activity but we planned that
some of these tasks would feed into other aspects of this project
(i.e.~scenario modelling). Participants discussed goals for breeding
waders and identified existing plans that and these discussion fed into
the scenario modelling part of the project. Participants also discussed
and ranked conservation strategies which was used as a primer for the
final task so stakeholders could discuss where their priorities lay
between the following conservation strategies: 1) improving existing
breeding wader sites (better); 2) expanding existing wader sites
(bigger); 3) creating new sites for breeding wader (more); and 4)
converting arable land to lowland wet grassland for breeding waders.
Although arable reversion is not a Lawton principle, it was defined as a
separate option here because stakeholder preferences for wet grassland
creation could markedly differ between existing unsuitable grassland and
arable land, i.e.~preferring reversion of arable land on peaty soils or
specific crop types.

Activity 3 generated the main outputs presented in this report and
stakeholder generally spent more time on this task than on the other two
activities combined. The purpose of this task was for stakeholders to
create preferences or avoidance guidelines for where breeding wader
conservation could or could not occur within the landscape. Preferences
were guidelines that could essentially grade the land into areas of
differing favorability, e.g.~prefer breeding wader conservation on lower
lying land. Avoidance rules mapped out where wader conservation would
not be carried out, e.g.~avoiding areas of priority habitat lowland fen.
Preferences could also be linked to one, multiple or all the
conservation strategies outlined in activity 2 (i.e.~better, bigger,
more, arable reversion). For example, preferring conservation efforts in
the smallest existing breeding populations first was linked to the
improving existing wader strategy (better), whereas preferring
conservation efforts on lower lying land was often associated with all
of the conservation strategies. All avoidance guidelines created applied
to all the conservation strategies. During stakeholder discussions there
was filtering of guidelines by the facilitator to remove any guidelines
that we would not be able to map out spatially. If there was any doubt,
then the guideline was recorded and if it could not be used then a full
explanation is provided in the appendix.

\subsection{Compiling Preferences}\label{compiling-preferences}

For each landscape and stakeholder group combination we produced heat
maps for the main conservation strategies, better, bigger and more. We
also produced heat maps for the bigger and more strategies being
realized through arable reversion which involved combing the guidelines
for arable reversion and bigger or more. In total, for any stakeholder
group this meant the creation of 5 different heat maps.

Each guideline was mapped out onto a 25m x 25m base raster. Each
preference guideline became a continuous raster with pixels given a
value between 0 (least preferred) and 1 (most preferred) and avoidance
rules became a binary raster of 0 (no avoidance) and 1 (avoid). Next,
for each 25m x 25m cell, preference rules were subsequently aggregated
(summed) to produce an overall preference score. Note, for simplicity,
individual preference rules were treated as equal with no form of
weighting applied. Last, any cells that was classified as 1 for any of
the avoidance rules were excluded from the opportunity area,
irrespective of their preference score. A 25m x 25m pixel size was
chosen as this is the resolution of the UKCEH land cover maps
(\citeproc{ref-marston2022}{Marston et al. 2022}) that was used to
identify areas of suitable grassland and arable land for lowland we
grassland creation (see following section). For the creation of each
graded raster including the manipulation, processing and analysis we
used the packages \emph{sf} (\citeproc{ref-pebesma2018}{Pebesma 2018})
and \emph{terra} (\citeproc{ref-hijmans2024}{Hijmans 2024}) in the
programming language R (\citeproc{ref-R2023}{R Core Team 2023}).

\subsection{Defining Conservation Strategy
Extent}\label{defining-conservation-strategy-extent}

For each conservation strategy there were only certain areas within the
landscapes where the strategy could be realized. For all strategies,
creation of lowland wet grassland for breeding waders had to be carried
out on land where this habitat could feasibly be created. Areas were
often unsuitable due to topography, land use of soil type. For better,
bigger and more strategies this land had to currently be some form of
grassland. When any of these strategies were realized using arable
reversion then the starting land use had to be arable. In addition, for
the strategy to improve existing wader sites (better) we had to define
where existing wader sites were within the landscape. Any areas outside
of defined wader sites could be used to expand existing sites (bigger)
or create more sites (more). We go into detail of how we define land
that has the right characteristics to be lowland wet grassland; current
arable land; and breeding wader sites below.

\paragraph{Defining candidate lowland wet grassland for
restoration}\label{defining-candidate-lowland-wet-grassland-for-restoration}

We defined current grassland that has the right characteristics,
i.e.~elevation and soil type, to be areas where high quality lowland wet
grassland could be created, regardless of its current condition.
Therefore, this included current high-quality lowland wet grassland as
well as dry grassland with drainage. This mapping exercise was done
using a base raster with a resolution of 25 meters. Potential lowland
wet grassland pixels included fields considered for survey from the
2021/2022 BWWM survey (\citeproc{ref-hawkes2024}{Hawes 2024}). These
fields were historically defined as periodically water-logged permanent
grassland below 200 meters above sea level, including grazing marshes,
flood meadows, man-made washlands, and water meadows. We supplemented
this with areas of semi-natural grassland habitats (Coastal and
floodplain grazing marsh; Good quality semi-improved grassland; Lowland
meadows; and Purple moor grass and rush pastures) from the Natural
England's priority habitat index (\citeproc{ref-natengland2022}{Natural
England 2022}). These supplementary areas also had to overlap with peaty
or seasonally wet soils from the NATMAP soil vector data (see
Table~\ref{tbl-wetsoil} for a full list of acceptable soil types
(\citeproc{ref-nsri2022}{NSRI 2022})) and be at an elevation below the
99.5th quantile of all elevation values within field included in the
2021/2022 BWWM survey (\citeproc{ref-hawkes2024}{Hawes 2024}). These
criteria prevented fields at high elevations being included. Since both
data sets were created more before the year of the study we masked out
any pixels classified as non-grassland habitats from any of the UKCEH
landcover datasets from 2021 (\citeproc{ref-marston2022}{Marston et al.
2022}), 2022 (\citeproc{ref-marston2024}{Marston et al. 2024}), or 2023
(\citeproc{ref-morton2024}{Morton et al. 2024}). Finally, we visually
checked every map to remove obvious arable land, woodland, salt marsh,
and golf courses.

In the Somerset Levels and Norfolk Broads in particular, small pockets
of trees were not detected in the UKCEH land cover data sets. To remove
all trees, we created a canopy model by subtracting the digital terrain
model (\citeproc{ref-envagency2023a}{Environment Agency 2023a}) from the
first pass digital surface model
(\citeproc{ref-envagency2023b}{Environment Agency 2023b}) from the
Environment Agency National Lidar Programme dataset. This canopy model
had a resolution of 1 meter, which we transformed to our base resolution
using nearest neighbor interpolation. We then used this layer to mask
out any pixels with a canopy height greater than 2 meters.

\begin{table}

\caption{\label{tbl-wetsoil}List of soil types from the NATMAP vector of
soil types that were to be seasonally wet or peaty soils}

\centering{

\centering\begingroup\fontsize{10}{12}\selectfont

\begin{tabular}[t]{l}
\toprule
\begingroup\fontsize{12}{14}\selectfont \textbf{Soil types}\endgroup\\
\midrule
\cellcolor{gray!10}{Fen peat soils}\\
\addlinespace
Lime-rich loamy and clayey soils with impeded drainage\\
\addlinespace
\cellcolor{gray!10}{Loamy and clayey floodplain soils with naturally high groundwater}\\
\addlinespace
Loamy and clayey soils of coastal flats with naturally high groundwater\\
\addlinespace
\cellcolor{gray!10}{Loamy and sandy soils with naturally high groundwater and a peaty surface}\\
\addlinespace
Loamy soils with naturally high groundwater\\
\addlinespace
\cellcolor{gray!10}{Naturally wet very acid sandy and loamy soils}\\
\addlinespace
Raised bog peat soils\\
\addlinespace
\cellcolor{gray!10}{Slightly acid loamy and clayey soils with impeded drainage}\\
\addlinespace
Slowly permeable seasonally wet acid loamy and clayey soils\\
\addlinespace
\cellcolor{gray!10}{Slowly permeable seasonally wet slightly acid but base-rich loamy and clayey soils}\\
\bottomrule
\end{tabular}
\endgroup{}

}

\end{table}%

\paragraph{Defining candidate arable land for
restoration}\label{defining-candidate-arable-land-for-restoration}

To identify arable land that could be reverted to high quality lowland
wet grassland we used a similar method to above. We first identified any
land that overlapped with peaty or seasonally wet soils from the NATMAP
soil vector data (\citeproc{ref-nsri2022}{NSRI 2022}). This land also
had to be at an elevation below the 99.5th quantile of all elevation
values within field included in the 2021/2022 BWWM survey
(\citeproc{ref-hawkes2024}{Hawes 2024}). We then identified arable
pixels as any that were identified as arable in at least two of the
UKCEH landcover datasets from 2021 (\citeproc{ref-marston2022}{Marston
et al. 2022}), 2022 (\citeproc{ref-marston2024}{Marston et al. 2024}),
or 2023 (\citeproc{ref-morton2024}{Morton et al. 2024}). Finally, we
visually checked every map to remove obvious woodland, salt marsh, and
golf courses and for the Somerset Levels and Norfolk Broads we used the
same tree mask as described above.

\paragraph{Defining existing wader
sites}\label{defining-existing-wader-sites}

There is no set definition of what an existing site for nature is, and
it could be quantified using habitat type, habitat quality or the
current distribution of species. We took the approach that existing
sites for breeding wader were the areas already occupied by breeding
waders. We used field-level data from the breeding waders of wet meadows
survey in 2021/22 and further gap filling surveys in 2023 to define
breeding wader sites. A field was defined as occupied if the number of
estimated breeding pairs of Lapwing, Redshank or Snipe in a field was
greater than 1 (see (\citeproc{ref-smart2014}{Smart et al. 2014}) for
pair estimation methods). We then created polygons around clusters of
occupied fields and defined these as breeding wader sites. This
clustering approach was used, instead of just using the occupied field
centroids, as it allowed us to capture suitable fields recorded as
unoccupied due to imperfect detection or factors other than habitat
quality. For example, some large reserves have large areas of suitable
habitat but not every single parcel is occupied by breeding waders. As
we did not carry out breeding surveys over the entire landscapes, this
approach could have missed some small breeding populations. However, for
all four landscapes, all sites with known wader populations were
surveyed unless access was denied. We consulted with regional
conservationist and recording bodies to confirm that the surveys did not
miss any previously known breeding populations. The full details of are
clustering approach are detailed below.

We used the centroids of all occupied fields to run a K-means clustering
analysis to identify distinct clusters of breeding wader fields within
each of the four landscapes (Figure~\ref{fig-kmeansclust}). This
essentially creates k number of clusters while minimizing the within
cluster sum of squares. We explored a range of different values for k
between 2 and 25 and chose the one at the elbow of the relationship
between within cluster sum of squares and k. For all regions we selected
a value of 6 or 7 for k. This analysis was done using the \emph{kmeans}
function in the R package stats (\citeproc{ref-R2023}{R Core Team
2023}). For each of the identified clusters we then create a kernel
density estimate using the field centroids in each cluster. We created
95\% kernel density estimate boundaries using the \emph{hr\_kde} in the
R package amt (\citeproc{ref-signer2019}{Signer, Fieberg, and Avgar
2019}) and set the bandwidth parameter to the median field width for
each landscape. The median field width ranged from 155.7m in the
Somerset Levels to 233.4m in Essex. Any overlapping boundaries between
clusters were combined at this stage. This step allowed smoothed
polygons to be created around clusters and was preferable to using
minimum convex polygons as often single occupied fields, far from
cluster centers, expended cluster polygons into unsuitable habitat. With
these smoothed polygons we classified individual land parcels as within
`wader sites' if they were more than 50\% covered and that the habitat
was been previously identified as suitable for lowland wet grassland.
Finally, we removed any small sites that contained very few land parcels
or had a very small population of waders that were too small to be a
viable population. Therefore, we removed clusters that contained three
or less land parcels or three or less pairs of breeding waders
(Figure~\ref{fig-finalclusters}).

\begin{figure}[H]

\centering{

\includegraphics{qmdimages/Som_KmeansClusts.png}

}

\caption{\label{fig-kmeansclust}Results of k-mean clustering of field
centroids for land parcels occupied by breeding wader in the Somerset
Levels. A value of 7 for k was chosen based off the elbow point of the
relationship between within cluster sum of squares and k}

\end{figure}%

\begin{figure}[H]

\centering{

\includegraphics{qmdimages/Som_Final_clusters.png}

}

\caption{\label{fig-finalclusters}Final retained 95\% kernel density
estimate polygons of breeding wader clusters for the Somerset Levels.
Each cluster was treated as a separate site}

\end{figure}%

\subsection{Linking this work to the wider
project}\label{linking-this-work-to-the-wider-project}

As part of the wider project we created different scenarios for wading
bird conservation in each landscape. These scenarios test which set of
conservation decisions lead to the most cost effective use of funding.
It tests the influence of the type of habitat management (AES vs
reserves), the quality of the management, whether a single large or
several small habitat patches are created, and where we create habitat
in relation to existing populations (i.e.~the `Lawton principles').
Using the findings from this analysis and the maps provided below we
hope that it is possible to work out the best conservation strategy for
each landscape and then using the heat maps identify priority areas for
habitat creation or restoration under the chosen strategy.

\section{Results}\label{results}

\subsection{North Kent Results}\label{north-kent-results}

For North Kent we had two different stakeholder groups. Group 1 was a
group of conservationists and group 2 was a group of landowners and
farmers. The stakeholder guidelines that were generated during the
workshops and how these were converted into a graded map can be found in
Table~\ref{tbl-KeG1} for group 1 and Table~\ref{tbl-KeG1} for group 2.
The fields identified as grassland or arable land that have the right
characteristics to be lowland wet grassland, regardless of their current
condition, can be seen in Figure~\ref{fig-KentSuitHab} and existing
wader sites can be seen in Figure~\ref{fig-KentLawton}.

\begin{figure}[H]

\centering{

\includegraphics[width=7.29167in,height=\textheight]{Plots/NorthKent_OpportunityHabitatMap.png}

}

\caption{\label{fig-KentSuitHab}Parcels in North Kent that were
identified as having the right soil type and elevation to become lowland
wet grassland, regardless of current condition. An OS map is used as the
background.}

\end{figure}%

\begin{figure}[H]

\centering{

\includegraphics[width=7.29167in,height=\textheight]{Plots/NorthKent_LawtonPrincipleMap.png}

}

\caption{\label{fig-KentLawton}Parcels in North Kent that we identified
as being part of lowland breeding wader clusters. These clusters were
identified using the breeding waders of wet meadows survey data from
2021-23.}

\end{figure}%

\newpage{}

\subsubsection{North Kent: Better}\label{north-kent-better}

The stakeholder preferences for the better principle of nature
restoration for group 1 and group 2 can be visualized in
(Figure~\ref{fig-NKBetterG1}) and (Figure~\ref{fig-NKBetterG2}),
receptively.

\begin{figure}[H]

\includegraphics[width=7.29167in,height=\textheight]{Plots/NorthKent_G1_Better.png}

\caption{\label{fig-NKBetterG1}Stakeholder gradings for group 1 in North
Kent for the better principle of nature restoration}

\end{figure}%

\begin{figure}[H]

\centering{

\includegraphics[width=7.29167in,height=\textheight]{Plots/NorthKent_G2_Better.png}

}

\caption{\label{fig-NKBetterG2}Stakeholder gradings for group 2 in North
Kent for the better principle of nature restoration}

\end{figure}%

\newpage{}

\subsubsection{North Kent: Bigger}\label{north-kent-bigger}

The stakeholder preferences for the bigger principle of nature
restoration for group 1 (Figure~\ref{fig-NKBigG1}) and 2
(Figure~\ref{fig-NKBigG2}).

\begin{figure}[H]

\centering{

\includegraphics[width=7.29167in,height=\textheight]{Plots/NorthKent_G1_Bigger.png}

}

\caption{\label{fig-NKBigG1}Stakeholder gradings for group 1 in North
Kent for the bigger principle of nature restoration}

\end{figure}%

\begin{figure}[H]

\centering{

\includegraphics[width=7.29167in,height=\textheight]{Plots/NorthKent_G2_Bigger.png}

}

\caption{\label{fig-NKBigG2}Stakeholder gradings for group 2 in North
Kent for the bigger principle of nature restoration}

\end{figure}%

\newpage{}

\subsubsection{North Kent: More}\label{north-kent-more}

The stakeholder preferences for the more principle of nature restoration
for group 1 (Figure~\ref{fig-NKMoreG1}) and 2
(Figure~\ref{fig-NKMoreG2}).

\begin{figure}[H]

\centering{

\includegraphics[width=7.29167in,height=\textheight]{Plots/NorthKent_G1_More.png}

}

\caption{\label{fig-NKMoreG1}Stakeholder gradings for group 1 in North
Kent for the more principle of nature restoration}

\end{figure}%

\begin{figure}[H]

\centering{

\includegraphics[width=7.29167in,height=\textheight]{Plots/NorthKent_G2_More.png}

}

\caption{\label{fig-NKMoreG2}Stakeholder gradings for group 2 in North
Kent for the more principle of nature restoration}

\end{figure}%

\newpage{}

\subsubsection{North Kent: Arable Reversion for
Bigger}\label{north-kent-arable-reversion-for-bigger}

The stakeholder preferences for the reversion of arable land to lowland
wet grassland under the bigger principle of nature restoration for group
1 (Figure~\ref{fig-NKArBigG1}) and group 2 (Figure~\ref{fig-NKArBigG2}).
Note: during the stakeholder workshops no guidelines or masks were
created specifically for arable reversion.

\begin{figure}[H]

\centering{

\includegraphics[width=7.29167in,height=\textheight]{Plots/NorthKent_G1_ArableBig.png}

}

\caption{\label{fig-NKArBigG1}Stakeholder gradings for group 1 in North
Kent for the reversion of arable land to lowland wet grassland under the
bigger principle of nature restoration}

\end{figure}%

\begin{figure}[H]

\centering{

\includegraphics[width=7.29167in,height=\textheight]{Plots/NorthKent_G2_ArableBig.png}

}

\caption{\label{fig-NKArBigG2}Stakeholder gradings for group 2 in North
Kent for the reversion of arable land to lowland wet grassland under the
bigger principle of nature restoration}

\end{figure}%

\newpage{}

\subsubsection{North Kent: Arable Reversion for
More}\label{north-kent-arable-reversion-for-more}

The stakeholder preferences for the reversion of arable land to lowland
wet grassland under the more principle of nature restoration for group 1
(Figure~\ref{fig-NKArMoreG1}) and group 2 (Figure~\ref{fig-NKArMoreG2}).
Note: during the stakeholder workshops no guidelines or masks were
created specifically for arable reversion.

\begin{figure}[H]

\centering{

\includegraphics[width=7.29167in,height=\textheight]{Plots/NorthKent_G1_ArableMore.png}

}

\caption{\label{fig-NKArMoreG1}Stakeholder gradings for group 1 in North
Kent for the reversion of arable land to lowland wet grassland under the
bigger principle of nature restoration}

\end{figure}%

\begin{figure}[H]

\centering{

\includegraphics[width=7.29167in,height=\textheight]{Plots/NorthKent_G2_ArableMore.png}

}

\caption{\label{fig-NKArMoreG2}Stakeholder gradings for group 2 in North
Kent for the reversion of arable land to lowland wet grassland under the
bigger principle of nature restoration}

\end{figure}%

\subsection{Acknowledgements}\label{acknowledgements}

We thank thank all the workshop attendees, and a specific thanks to
Damon Bridge, Kieran Alexander, Will Tofts, Ian Robinson, Alan Johnson
and Mark Smart for helping to organise these workshops. We also thank
Malcolm Ausden and the wider lowland wader scenario steering group team
for input into the project. We also need to thank NE for funding.

\newpage{}

\subsection*{Bibliography}\label{bibliography}
\addcontentsline{toc}{subsection}{Bibliography}

\phantomsection\label{refs}
\begin{CSLReferences}{1}{0}
\bibitem[\citeproctext]{ref-chazdon2016}
Chazdon, R. L., P. H. S. Brancalion, D. Lamb, L. Laestadius, M. Calmon,
and C. Kumar. 2016. {``A Policy{-}Driven Knowledge Agenda for Global
Forest and Landscape Restoration.''} \emph{Conservation Letters} 10 (1):
125--32. \url{https://doi.org/10.1111/conl.12220}.

\bibitem[\citeproctext]{ref-donald2001}
Donald, P. F., R. E. Green, and M. F. Heath. 2001. {``Agricultural
Intensification and the Collapse of Europe's Farmland Bird
Populations.''} \emph{Proceedings of the Royal Society of London. Series
B: Biological Sciences} 268 (1462): 25--29.
\url{https://doi.org/10.1098/rspb.2000.1325}.

\bibitem[\citeproctext]{ref-eglington2010}
Eglington, S. M., M. Bolton, M. A. Smart, W. J. Sutherland, A. R.
Watkinson, and J. A. Gill. 2010. {``Managing Water Levels on Wet
Grasslands to Improve Foraging Conditions for Breeding Northern
Lapwing{\emph{Vanellus Vanellus}}.''} \emph{Journal of Applied Ecology}
47 (2): 451--58. \url{https://doi.org/10.1111/j.1365-2664.2010.01783.x}.

\bibitem[\citeproctext]{ref-envagency2023a}
Environment Agency. 2023a. \emph{LIDAR Composite Digital Terrain Model
(DTM) - 1m}.
\url{https://environment.data.gov.uk/dataset/13787b9a-26a4-4775-8523-806d13af58fc}.

\bibitem[\citeproctext]{ref-envagency2023b}
---------. 2023b. \emph{LIDAR Composite First Return Digital Surface
Model (FZ\_DSM) - 1m}.
\url{https://environment.data.gov.uk/dataset/df4e3ec3-315e-48aa-aaaf-b5ae74d7b2bb}.

\bibitem[\citeproctext]{ref-franks2018}
Franks, S. E., M. Roodbergen, W. Teunissen, Carrington. C. A., and J. W.
Pearce-Higgins. 2018. {``Evaluating the Effectiveness of Conservation
Measures for European Grassland{-}Breeding Waders.''} \emph{Ecology and
Evolution} 8 (21): 10555--68. \url{https://doi.org/10.1002/ece3.4532}.

\bibitem[\citeproctext]{ref-gamborg2019}
Gamborg, C., J. Morsing, and K. Raulund-Rasmussen. 2019. {``Adjustive
Ecological Restoration Through Stakeholder Involvement: A Case of
Riparian Landscape Restoration on Privately Owned Land with Public
Access.''} \emph{Restoration Ecology} 27 (5): 1073--83.
\url{https://doi.org/10.1111/rec.12955}.

\bibitem[\citeproctext]{ref-hawkes2024}
Hawes, R. W. 2024. {``Landscape Context Influences Efficacy of Protected
Areas and Agri-Environment Scheme Delivery for Waders Breeding in
Lowland Wet Grassland.''} \emph{In Prep}.

\bibitem[\citeproctext]{ref-hijmans2024}
Hijmans, R. J. 2024. \emph{Terra: Spatial Data Analysis}.
\url{https://CRAN.R-project.org/package=terra}.

\bibitem[\citeproctext]{ref-jackson1994}
Jackson, D. B. 1994. {``Breeding Dispersal and Site{-}Fidelity in Three
Monogamous Wader Species in the Western Isles, U.K.''} \emph{Ibis} 136
(4): 463--73. \url{https://doi.org/10.1111/j.1474-919x.1994.tb01123.x}.

\bibitem[\citeproctext]{ref-lawton2010}
Lawton, J. H., P. N. M. Brotherton, V. K. Brown, C. Elphick, A. H.
Fitter, J. Forshaw, R. W. Haddow, et al. 2010. {``Making Space for
Nature: A Review of England's Wildlife Sites and Ecological Network.
Report to Defra.''}

\bibitem[\citeproctext]{ref-malpas2013}
Malpas, L. R., R. J. Kennerley, Graham J. M. Hirons, R. D. Sheldon, M.
Ausden, J. C. Gilbert, and J. Smart. 2013. {``The Use of
Predator-Exclusion Fencing as a Management Tool Improves the Breeding
Success of Waders on Lowland Wet Grassland.''} \emph{Journal for Nature
Conservation} 21 (1): 37--47.
\url{https://doi.org/10.1016/j.jnc.2012.09.002}.

\bibitem[\citeproctext]{ref-marston2024}
Marston, C. G., R. D. Morton, A. W. O'Neil, and C. S. Rowland. 2024.
{``Land Cover Map 2022 (25m Rasterised Land Parcels, GB).''} NERC EDS
Environmental Information Data Centre.
\url{https://doi.org/10.5285/C9449BF5-B8F6-4A1C-B3EB-0D70575CBA39}.

\bibitem[\citeproctext]{ref-marston2022}
Marston, C. G., C. S. Rowland, A. W. O'Neil, and R. D. Morton. 2022.
{``Land Cover Map 2021 (25m Rasterised Land Parcels, GB).''} NERC EDS
Environmental Information Data Centre.
\url{https://doi.org/10.5285/A1F85307-CAD7-4E32-A445-84410EFDFA70}.

\bibitem[\citeproctext]{ref-morton2024}
Morton, R. D., C. G. Marston, A. W. O'Neil, and C. S. Rowland. 2024.
{``Land Cover Map 2023 (25m Rasterised Land Parcels, GB).''} NERC EDS
Environmental Information Data Centre.
\url{https://doi.org/10.5285/AB10EA4A-1788-4D25-A6DF-F1AFF829DFFF}.

\bibitem[\citeproctext]{ref-natengland2022}
Natural England. 2022. \emph{Priority Habitats Inventory (England)}.
\url{https://www.data.gov.uk/dataset/4b6ddab7-6c0f-4407-946e-d6499f19fcde/priority-habitats-inventory-england}.

\bibitem[\citeproctext]{ref-nateng2024}
---------. 2024. \emph{Environmentally Sensitive Areas (England)}.
\href{https://www.data.gov.uk/\%20dataset/a5b0ccc4-a144-4027-91fa-49084ff07da2/environmentally-sensitive-areas-england\%20}{https://www.data.gov.uk/
dataset/a5b0ccc4-a144-4027-91fa-49084ff07da2/environmentally-sensitive-areas-england
}.

\bibitem[\citeproctext]{ref-nsri2022}
NSRI. 2022. \emph{National Soil Map of England and Wales - NATMAP
Vector, 1:250000 Soil Association Map}.
\href{https://www.landis.org.uk/data/natmap.cfm\%20}{https://www.landis.org.uk/data/natmap.cfm
}.

\bibitem[\citeproctext]{ref-pebesma2018}
Pebesma, E. 2018. {``{Simple Features for R: Standardized Support for
Spatial Vector Data}.''} \emph{{The R Journal}} 10 (1): 439--46.
\url{https://doi.org/10.32614/RJ-2018-009}.

\bibitem[\citeproctext]{ref-R2023}
R Core Team. 2023. \emph{R: A Language and Environment for Statistical
Computing}. R Foundation for Statistical Computing.
\url{https://www.R-project.org/}.

\bibitem[\citeproctext]{ref-raven2021}
Raven, P. H., and D. L. Wagner. 2021. {``Agricultural Intensification
and Climate Change Are Rapidly Decreasing Insect Biodiversity.''}
\emph{Proceedings of the National Academy of Sciences} 118 (2).
\url{https://doi.org/10.1073/pnas.2002548117}.

\bibitem[\citeproctext]{ref-roodbergen2011}
Roodbergen, M., B. van der Werf, and H. Hötker. 2011. {``Revealing the
Contributions of Reproduction and Survival to the Europe-Wide Decline in
Meadow Birds: Review and Meta-Analysis.''} \emph{Journal of Ornithology}
153 (1): 53--74. \url{https://doi.org/10.1007/s10336-011-0733-y}.

\bibitem[\citeproctext]{ref-roos2018}
Roos, St.n, J. Smart, D. W. Gibbons, and J. D. Wilson. 2018. {``A Review
of Predation as a Limiting Factor for Bird Populations in
Mesopredator{-}Rich Landscapes: A Case Study of the UK.''}
\emph{Biological Reviews} 93 (4): 1915--37.
\url{https://doi.org/10.1111/brv.12426}.

\bibitem[\citeproctext]{ref-sabatier2010}
Sabatier, R., L. Doyen, and M. Tichit. 2010. {``Modelling Trade-Offs
Between Livestock Grazing and Wader Conservation in a Grassland
Agroecosystem.''} \emph{Ecological Modelling} 221 (9): 1292--1300.
\url{https://doi.org/10.1016/j.ecolmodel.2010.02.003}.

\bibitem[\citeproctext]{ref-signer2019}
Signer, J., J. Fieberg, and T. Avgar. 2019. {``Animal Movement Tools
(Amt): R Package for Managing Tracking Data and Conducting Habitat
Selection Analyses.''} \emph{Ecology and Evolution} 9 (2): 880--90.
\url{https://doi.org/10.1002/ece3.4823}.

\bibitem[\citeproctext]{ref-smart2014}
Smart, J., S. R. Wotton, I. A. Dillon, A. I. Cooke, I. Diack, A. L.
Drewitt, P. V. Grice, and R. D. Gregory. 2014. {``Synergies Between Site
Protection and Agri{-}Environment Schemes for the Conservation of Waders
on Lowland Wet Grasslands.''} \emph{Ibis} 156 (3): 576--90.
\url{https://doi.org/10.1111/ibi.12153}.

\bibitem[\citeproctext]{ref-song2018}
Song, X., M. C. Hansen, S. V. Stehman, P. V. Potapov, A. Tyukavina, E.
F. Vermote, and J. R. Townshend. 2018. {``Global Land Change from 1982
to 2016.''} \emph{Nature} 560 (7720): 639--43.
\url{https://doi.org/10.1038/s41586-018-0411-9}.

\bibitem[\citeproctext]{ref-verhulst2011}
Verhulst, J., D. Kleijn, W. Loonen, F. Berendse, and C. Smit. 2011.
{``Seasonal Distribution of Meadow Birds in Relation to in-Field
Heterogeneity and Management.''} \emph{Agriculture, Ecosystems \&
Environment} 142 (3-4): 161--66.
\url{https://doi.org/10.1016/j.agee.2011.04.016}.

\bibitem[\citeproctext]{ref-wilson2005}
Wilson, A. M., J. A. Vickery, A. Brown, R. H. W. Langston, D.
Smallshire, S. Wotton, and D. Vanhinsbergh. 2005. {``Changes in the
Numbers of Breeding Waders on Lowland Wet Grasslands in England and
Wales Between 1982 and 2002.''} \emph{Bird Study} 52 (1): 55--69.
\url{https://doi.org/10.1080/00063650509461374}.

\end{CSLReferences}

\newpage{}

\subsection{Appendix}\label{appendix}

\beginsupplement

\newpage{}

\begingroup\fontsize{7}{9}\selectfont

\begin{longtable}[t]{>{\raggedright\arraybackslash}p{5em}|>{\raggedright\arraybackslash}p{10em}|>{\raggedright\arraybackslash}p{15em}|>{\raggedright\arraybackslash}p{30em}}

\caption{\label{tbl-KeG1}Stakeholder rules generated during a workshop
for North Kent group 1 (conservationists). See Table~\ref{tbl-Citations}
for full citation of each reference}

\tabularnewline

\hline
\begingroup\fontsize{8}{10}\selectfont \textbf{Strategies}\endgroup & \begingroup\fontsize{8}{10}\selectfont \textbf{Guideline}\endgroup & \begingroup\fontsize{8}{10}\selectfont \textbf{Reference}\endgroup & \begingroup\fontsize{8}{10}\selectfont \textbf{Implementation}\endgroup\\
\hline
\endfirsthead
\multicolumn{4}{@{}l}{\textit{(continued)}}\\
\hline
\begingroup\fontsize{8}{10}\selectfont \textbf{Strategies}\endgroup & \begingroup\fontsize{8}{10}\selectfont \textbf{Guideline}\endgroup & \begingroup\fontsize{8}{10}\selectfont \textbf{Reference}\endgroup & \begingroup\fontsize{8}{10}\selectfont \textbf{Implementation}\endgroup\\
\hline
\endhead
\cellcolor{gray!10}{better} & \cellcolor{gray!10}{Target smallest existing populations} & \cellcolor{gray!10}{New data set} & \cellcolor{gray!10}{Within identified wader clusters all breeding pairs of lapwing and redshank are summed. The population sizes are then scaled so that clusters with lowest total population receives a score of 1 and the highest population a score of 0.}\\
\hline
better & Target areas with continuity in management & NA & **NOT USED** Was not sure what this really meant, could not work out how we would even map this during the workshop. Have left it in here to show that it was an idea but that it could not be included.\\
\hline
\cellcolor{gray!10}{better} & \cellcolor{gray!10}{Avoid RSPB reserves and Elmley NNR} & \cellcolor{gray!10}{New data set} & \cellcolor{gray!10}{Select RSPB reserves with more than 50 pairs and Elmley NNR. All pixels outside of these areas get a grading of 1 and inside these reserves a grading of 0.}\\
\hline
bigger & Target areas that link existing wader sites & New data set; New data set (see methods) & Least cost path analysis was used for this rule. The resistance surface had the following values based upon land use: opportunity lowland grassland = 5; opportunity arable = 3; and all other habitats 1). This surface is used to calculate the least costs paths between the centroid of all wader sites in the landscape. Next, I calculate the number of least costs paths that pass through a 2km resolution raster of the priority landscape and scale pixel values. Therefore, the pixels with the greatest number of paths have a value of 1. Finally, this 2km raster is converted back to a 25m raster using smoothing.\\
\hline
\cellcolor{gray!10}{more} & \cellcolor{gray!10}{Target areas that link most isolated sites from other sites} & \cellcolor{gray!10}{New data set (see methods)} & \cellcolor{gray!10}{Least cost path analysis was used for this rule. The resistance surface had the following values based upon land use: opportunity lowland grassland = 5; opportunity arable = 3; and all other habitats 1). I then calculate the least cost paths between the centroid of all wader sites in the landscapes. For each wader site I sum the total least cost path length (so sites that are more isolated have a higher value) and use these totals to weight each least cost paths. Then I calculate the total value of all weighted least costs paths that pass through each pixel of a 2km resolution raster. This 2km raster is then scaled so the pixels with the highest values are graded 1. Finally, I turn this 2km raster back to a 25m raster to smooth it.}\\
\hline
better/ bigger & Target sites that are near the biggest wader sites & New data set & First filter out any wader sites that have a total wader population of greater than 50 pairs. These become our “large wader sites”. Then calculate the distance between all pixels and the nearest large wader sites. Finally, I take the inverse of all the distances so that the pixels in or immediately on the border of a large wader sites have a grading of 1.\\
\hline
\cellcolor{gray!10}{bigger/ more} & \cellcolor{gray!10}{Target fields with lower public footfall} & \cellcolor{gray!10}{(Day \& Smith 2018)} & \cellcolor{gray!10}{Waiting for OrVAL to come back online had been down this last week}\\
\hline
bigger/ more & Target fields further from urban areas & (Marston et al. 2022) & I used the UKCEH landcover data to determine where urban areas are in the landscape. Then I used the focal function with a 1.025 km square focal window to smooth a 25 x 25m raster of urban landcover. Finally, I take inverse of the values in this smoothed raster so areas with no urban are graded 1 and areas with the most urban in a 1.025km box = 0.\\
\hline
\cellcolor{gray!10}{bigger/ more} & \cellcolor{gray!10}{Target field further from biofuel production sites} & \cellcolor{gray!10}{NA} & \cellcolor{gray!10}{**NOT USED** Could not find any data sets that mapped out biofuel production sites.}\\
\hline
bigger/ more & Target fields with lowest agricultural land gradings & (Natural England 2019) & The land with the highest grading (grade 4 and none-agricultural) are given a grading of 1, land with the highest grading (grade 1) as well as urban areas are given a value of 0. Intermediate land gradings are given intermediate raster gradings.\\
\hline
\cellcolor{gray!10}{bigger/ more} & \cellcolor{gray!10}{Target areas that are not targeted for saltmarsh creation} & \cellcolor{gray!10}{(RSPB 2018)} & \cellcolor{gray!10}{This guideline used the sustainable shores project data set. This project mapped out areas for habitat creation around the UK coastline, including saltmarsh. This is split into sites that are opportunity areas and then further classifies these opportunity sites as whether they are a priority or not. Priority sites have a grading of 0, opportunity sites a grading of 0.5 and all other areas get a grading of 0.}\\
\hline
bigger/ more & Target old grassland fields over improved/reseeded fields & (Rowland et al. 2020); (Fuller et al. 2022); (Morton et al. 2014) & Using the UKCEH landcover data sets for 2000 and 1990 I identified fields that used to be grassland in these years. If a field was grassland in both years, it was given a grading of 1, on only one year 0.5 and in neither year a grading of 0.\\
\hline
\cellcolor{gray!10}{all} & \cellcolor{gray!10}{Target areas with more water in the future} & \cellcolor{gray!10}{(Environment Agency 2024a); (Environment Agency 2013a); (Environment Agency 2013b)} & \cellcolor{gray!10}{For each EA catchment zone there is a rough percentage of time that water is available for abstraction. These catchments are polygons that are then rasterized. The largest percentage of time that water is available for abstraction is graded 1 and the lowest percentage time is graded 0.}\\
\hline
all & Avoid highest areas of Hoo peninsula & NA & **NOT USED** Looking at the elevation map of the Hoo Peninsula all the higher areas in the middle are not included in the priority landscape so all these areas have already been removed.\\
\hline
\cellcolor{gray!10}{all} & \cellcolor{gray!10}{Avoid areas near large heronries} & \cellcolor{gray!10}{New data set} & \cellcolor{gray!10}{Identify all pixels within a 2km buffer of Grey Heron colony centroids. Then calculate the distance of all pixels within this 2km buffer from the centroid. Then scale all the values in the buffer so those 2km away from the colony have a grading of 1 and those in the centre a grading of 0. All other pixels outside the buffer have value of 1.}\\
\hline
all & Avoid areas with sandy and gravel soils & NA & **NOT USED** Sandy/gravel soils tend to be freely draining so this rule has already been applied for all regions as we only allow wet and impeded drainage soils for lowland wet grassland creation from grassland or arable.\\
\hline
\cellcolor{gray!10}{all} & \cellcolor{gray!10}{Avoid scheduled monuments} & \cellcolor{gray!10}{(Historic England 2024)} & \cellcolor{gray!10}{All pixels that overlap a scheduled monument polygon (+ 20m buffer) by more than 50\% are masked out. This buffer is based on recommendations from Natural Heritage.}\\
\hline
all & Mask urban areas & (Marston et al. 2022) & All pixels in the UKCEH habitat data that are assigned as urban/suburban, or a coastal habitat are turned into masks. As the UKCEH 25m raster is the base for all masks this is simply selecting certain pixels.\\
\hline
\cellcolor{gray!10}{all} & \cellcolor{gray!10}{Avoid priority habitats} & \cellcolor{gray!10}{(Natural England 2024a)} & \cellcolor{gray!10}{All pixels that overlap a non-lowland wet grassland priority habitat polygon by more than 50\% are assigned as a masked pixel. This includes priority habitat woodland, raised bog, dry grasslands, heathland, reedbed and fen.}\\
\hline

\end{longtable}

\endgroup{}

\newpage{}

\begingroup\fontsize{7}{9}\selectfont

\begin{longtable}[t]{>{\raggedright\arraybackslash}p{5em}|>{\raggedright\arraybackslash}p{10em}|>{\raggedright\arraybackslash}p{15em}|>{\raggedright\arraybackslash}p{30em}}

\caption{\label{tbl-KeG2}Stakeholder rules generated during a workshop
for North Kent group 2 (landowners and famers). See
Table~\ref{tbl-Citations} for full citation of each reference}

\tabularnewline

\hline
\begingroup\fontsize{8}{10}\selectfont \textbf{Strategies}\endgroup & \begingroup\fontsize{8}{10}\selectfont \textbf{Guideline}\endgroup & \begingroup\fontsize{8}{10}\selectfont \textbf{Reference}\endgroup & \begingroup\fontsize{8}{10}\selectfont \textbf{Implementation}\endgroup\\
\hline
\endfirsthead
\multicolumn{4}{@{}l}{\textit{(continued)}}\\
\hline
\begingroup\fontsize{8}{10}\selectfont \textbf{Strategies}\endgroup & \begingroup\fontsize{8}{10}\selectfont \textbf{Guideline}\endgroup & \begingroup\fontsize{8}{10}\selectfont \textbf{Reference}\endgroup & \begingroup\fontsize{8}{10}\selectfont \textbf{Implementation}\endgroup\\
\hline
\endhead
\cellcolor{gray!10}{better} & \cellcolor{gray!10}{Target smallest existing populations} & \cellcolor{gray!10}{New data set} & \cellcolor{gray!10}{Within identified wader clusters all breeding pairs of lapwing and redshank are summed. The population sizes are then scaled so that clusters with lowest total population receives a score of 1 and the highest population a score of 0.}\\
\hline
bigger & Target sites that are near the biggest wader sites & New data set & First filter out any wader sites that have a total wader population of greater than 50 pairs. These become our “large wader sites”. Then calculate the distance between all pixels and the nearest large wader sites. Finally, I take the inverse of all the distances so that the pixels in or immediately on the border of a large wader sites have a grading of 1.\\
\hline
\cellcolor{gray!10}{more} & \cellcolor{gray!10}{Target areas in between existing wader sites} & \cellcolor{gray!10}{New data set} & \cellcolor{gray!10}{Calculate the distance between all pixels and the nearest wader site of any size. Then take the inverse of all the distances so that the pixels in or immediately on the border of a large wader sites have a grading of 0 and those the furthest from wader sites are given a grading of 0.}\\
\hline
bigger/ more & Target fields with lower public footfall & (Day \& Smith 2018) & Waiting for OrVAL to come back online had been down this last week\\
\hline
\cellcolor{gray!10}{bigger/ more} & \cellcolor{gray!10}{Target areas close to where reservoirs can be built} & \cellcolor{gray!10}{NA} & \cellcolor{gray!10}{**NOT USED** Hard to map out where water storage will be in the future. Could potentially go anywhere so does not make certain areas more of a priority than others. Also, could not find any data that would map out where water storage reservoirs would go.}\\
\hline
bigger/ more & Target areas that already had wader CS schemes history & (Natural England 2024c); (Rural Payments Agency 2024) & I identified land parcels that had either wader specific CSS or ESS agreements and then rasterized these (this included wintering and breeding wader agreements. The codes used for CSS were: GS9, GS10, GS11 and GS12 and ESS were: HK9, HK11, HK13, HK10, HK12, and HK14. AES field graded 1 and all other areas graded 0.\\
\hline
\cellcolor{gray!10}{bigger/ more} & \cellcolor{gray!10}{Target fields further from urban areas} & \cellcolor{gray!10}{(Natural England 2024a)} & \cellcolor{gray!10}{I used the UKCEH landcover data to determine where urban areas are in the landscape. Then I used the focal function with a 1.025 km square focal window to smooth a 25 x 25m raster of urban landcover. Finally, I take inverse of the values in this smoothed raster so areas with no urban are graded 1 and areas with the most urban in a 1.025km box = 0.}\\
\hline
all & Target low lying fields & (Environment Agency 2022a) & Using a 2m elevation map of the landscape I extracted the elevation values within each hydrological unit. Using these I computed an empirical cumulative distribution function and assigned quantile values to all elevation values in the unit and then took the inverse of these values. Therefore, within each hydro unit the highest areas have a value of 0 and the lowest of 1. Not all pixels within the landscape were within a mapped out hydrological unit. For these pixels I created an empirical cumulative distribution function based of elevation values in all hydro units and then calculated the quantile for all elevation values outside of hydro units but within the priority landscape (ensuring to take the inverse of these quantile values again).\\
\hline
\cellcolor{gray!10}{all} & \cellcolor{gray!10}{Avoid scheduled monuments} & \cellcolor{gray!10}{(Historic England 2024)} & \cellcolor{gray!10}{All pixels that overlap a scheduled monument polygon (+ 20m buffer) by more than 50\% are masked out. This buffer is based on recommendations from Natural Heritage.}\\
\hline
all & Maks out urban areas & (Marston et al. 2022) & All pixels in the UKCEH habitat data that are assigned as urban/suburban, or a coastal habitat are turned into masks. As the UKCEH 25m raster is the base for all masks this is simply selecting certain pixels.\\
\hline
\cellcolor{gray!10}{all} & \cellcolor{gray!10}{Avoid priority habitats} & \cellcolor{gray!10}{(Natural England 2024a)} & \cellcolor{gray!10}{All pixels that overlap a non-lowland wet grassland priority habitat polygon by more than 50\% are assigned as a masked pixel. This includes priority habitat woodland, raised bog, dry grasslands, heathland, reedbed and fen.}\\
\hline

\end{longtable}

\endgroup{}

\newpage{}

\begingroup\fontsize{7}{9}\selectfont

\begin{longtable}[t]{>{\raggedright\arraybackslash}p{10em}|>{\raggedright\arraybackslash}p{50em}}

\caption{\label{tbl-Citations}A full list of citations that were used to
turn the stakeholder guildelines into realy world spatially graded
layers}

\tabularnewline

\hline
\begingroup\fontsize{8}{10}\selectfont \textbf{Reference}\endgroup & \begingroup\fontsize{8}{10}\selectfont \textbf{Dataset References}\endgroup\\
\hline
\endfirsthead
\multicolumn{2}{@{}l}{\textit{(continued)}}\\
\hline
\begingroup\fontsize{8}{10}\selectfont \textbf{Reference}\endgroup & \begingroup\fontsize{8}{10}\selectfont \textbf{Dataset References}\endgroup\\
\hline
\endhead
\cellcolor{gray!10}{(Borrelli et al. 2017)} & \cellcolor{gray!10}{Borrelli, P., Lugato, E., Montanarella, L., \& Panagos, P. (2017). A New Assessment of Soil Loss Due to Wind Erosion in European Agricultural Soils Using a Quantitative Spatially Distributed Modelling Approach. Land Degradation \& Development, 28: 335–344, DOI: 10.1002/ldr.2588, DOI: 10.1002/ldr.2588}\\
\hline
(Day \& Smith 2018) & Day, B., \& Smith, G. S. (2018). The Outdoor Recreation Valuation (ORVal) tool: Technical report, January 2018. Report to the Department of Food and Rural Afairs.\\
\hline
\cellcolor{gray!10}{(Environment Agency 2013a)} & \cellcolor{gray!10}{Environmental Agency. (2013). North Kent \& Swale Abstraction Licensing Strategy February 2013. Date Accessed: 01-07-2024. URL: https://www.gov.uk/government/publications/north-kent-and-swale-catchment-abstraction-licensing-strategy}\\
\hline
(Environment Agency 2013b) & Environmental Agency. (2013). Medway Abstraction licensing strategy February 2013. Date Accessed: 01-07-2024. URL: https://www.gov.uk/government/publications/medway-catchment-abstraction-licensing-strategy\\
\hline
\cellcolor{gray!10}{(Environment Agency 2013c)} & \cellcolor{gray!10}{Environmental Agency. (2013). Essex abstraction licensing strategy. Date Accessed: 01-07-2024. URL: https://www.gov.uk/government/publications/cams-essex-abstraction-licensing-strategy}\\
\hline
(Environment Agency 2022a) & Environment Agancy. (2022). LIDAR Composite Digital Terrain Model (DTM) 2m. Date Accessed: 01-06-2024. URL: https://environment.data.gov.uk/dataset/09ea3b37-df3a-4e8b-ac69-fb0842227b04\\
\hline
\cellcolor{gray!10}{(Environment Agency 2022b)} & \cellcolor{gray!10}{Environment Agancy. (2022). South and West Somerset abstraction licensing strategy. Date Accessed: 29-08-2024. URL: https://www.gov.uk/government/publications/south-and-west-somerset-abstraction-licensing-strategy/south-and-west-somerset-abstraction-licensing-strategy}\\
\hline
(Environment Agency 2024a) & Environmental Agency. (2024). Water Resource Availability and Abstraction Reliability Cycle 2. Date Accessed: 01-07-2024. URL: https://environment.data.gov.uk/dataset/62514eb5-e9d5-4d96-8b73-a40c5b702d43\\
\hline
\cellcolor{gray!10}{(Environment Agency 2024b)} & \cellcolor{gray!10}{Environment Agency. (2024). Permitted Waste Sites - Authorised Landfill Site Boundaries. Date Accessed: 01-07-2024. URL: https://www.data.gov.uk/dataset/ad695596-d71d-4cbb-8e32-99108371c0ee/permitted-waste-sites-authorised-landfill-site-boundaries}\\
\hline
(Fuller et al. 2022) & Fuller, R.M.; Smith, G.M.; Sanderson J.M.; Hill, R.A.; Thomson, A.G; Cox, R.; Brown, N.J.; Clarke, R.T; Rothery, P.; Gerard, F.F. (2002). Land Cover Map 2000 (25m raster, GB). NERC Environmental Information Data Centre. https://doi.org/10.5285/f802edfc-86b7-4ab9-b8fa-87e9135237c9\\
\hline
\cellcolor{gray!10}{(Historic England 2024)} & \cellcolor{gray!10}{Historic England. (2024). Scheduled Monuments. Date Accessed: 28-08-2024. URL: https://opendata-historicengland.hub.arcgis.com/datasets/historicengland::national-heritage-list-for-england-nhle/explore?layer=6}\\
\hline
(Holm \& Laursen 2009) & Holm, T.E. and Laursen, K. (2009). Experimental disturbance by walkers affects behaviour and territory density of nesting Black-tailed GodwitLimosa limosa. Ibis, 151(1), pp.77–87. doi:https://doi.org/10.1111/j.1474-919x.2008.00889.x.\\
\hline
\cellcolor{gray!10}{(Lislevand et al. 2009)} & \cellcolor{gray!10}{Lislevand, T., Byrkjedal, I. and Grønstøl, G. B. (2009) “Dispersal and age at first breeding in Norwegian Northern Lapwings (Vanellus vanellus)”, Ornis Fennica, 86(1), pp. 11–17. Available at: https://ornisfennica.journal.fi/article/view/133716}\\
\hline
(Marston et al. 2022) & Marston, C.; Rowland, C.S.; O’Neil, A.W.; Morton, R.D. (2022). Land Cover Map 2021 (25m rasterised land parcels, GB). NERC EDS Environmental Information Data Centre. https://doi.org/10.5285/a1f85307-cad7-4e32-a445-84410efdfa70\\
\hline
\cellcolor{gray!10}{(Morton et al. 2014)} & \cellcolor{gray!10}{Morton, R.D.; Rowland, C.S.; Wood, C.M.; Meek, L.; Marston, C.G.; Smith, G.M. (2014). Land Cover Map 2007 (25m raster, GB) v1.2. NERC Environmental Information Data Centre. https://doi.org/10.5285/a1f88807-4826-44bc-994d-a902da5119c2}\\
\hline
(Natural England 2019) & Natural England. (2019). Provisional agricultural land classification (ALC) natural England, United Kingdom. Date Accessed: 27-08-2024. URL: https://naturalen gland-defra.opendata.arcgis.com/datasets/5d2477d8d04b41d4bbc9a8742f858f4d \_0.\\
\hline
\cellcolor{gray!10}{(Natural England 2020a)} & \cellcolor{gray!10}{Natural England. (2020). Habitat Networks (England) - Lowland Fen. Date Accessed: 28-08-2024. URL: https://environment.data.gov.uk/dataset/1c85a398-653a-4a21-9923-f5d09adfea3a}\\
\hline
(Natural England 2020b) & Natural England. (2020). Habitat Networks (England) - Reedbed. Date Accessed: 28-08-2024. URL: https://environment.data.gov.uk/dataset/4b93c91b-3c7f-4ad2-9fe7-ad93e920b1ad;\\
\hline
\cellcolor{gray!10}{(Natural England 2020c)} & \cellcolor{gray!10}{Natural England. (2020). Habitat Networks (England) - Lowland Raised Bog. Date Accessed: 28/08-2024. URL: https://environment.data.gov.uk/dataset/c8244a2d-6e53-499a-8419-b41aae88a90e}\\
\hline
(Natural England 2024a) & Natural England. (2024). Priority Habitats Inventory (England). Date Accessed: 28-08-2024. URL:https://naturalengland-defra.opendata.arcgis.com/datasets/Defra::priority-habitats-inventory-england/about\\
\hline
\cellcolor{gray!10}{(Natural England 2024b)} & \cellcolor{gray!10}{Natural England. (2024). Sites of Special Scientific Interest (England). Date Accessed: 28-08-2024. URL: https://naturalengland-defra.opendata.arcgis.com/datasets/f10cbb4425154bfda349ccf493487a80\_0/about}\\
\hline
(Natural England 2024c) & Natural England. (2024). Countryside Stewardship Scheme Options (England). Date Accessed: 01-05-2024. URL: https://naturalengland-defra.opendata.arcgis.com/datasets/Defra::countryside-stewardship-scheme-options-england/about\\
\hline
\cellcolor{gray!10}{(Natural England 2024d)} & \cellcolor{gray!10}{Natural England. (2024). King Charles III England Coast Path Route. Date Accessed: 25-07-2024. URL: https://naturalengland-defra.opendata.arcgis.com/datasets/a1488f928832407fbd267feb6802bed6\_0/about}\\
\hline
(NSRI 2022) & NSRI (2022) National soil map of England and Wales - NATMAP Vector, 1:250000 Soil Association Map.  Date Accessed: 01-02-2024. URL: https://www.landis.org.uk/data/natmap.cfm.\\
\hline
\cellcolor{gray!10}{(Ordanance Survey 2024)} & \cellcolor{gray!10}{Ordanance Survey. (2024). OS Open Rivers. Date Accessed: 01-06-2024. URL: https://www.data.gov.uk/dataset/dc29160b-b163-4c6e-8817-f313229bcc23/os-open-rivers}\\
\hline
(Panagos et al. 2015) & Panagos, P., Borrelli, P., Poesen, J., Ballabio, C., Lugato, E., Meusburger, K., Montanarella, L., Alewell, .C. (2015). The new assessment of soil loss by water erosion in Europe. Environmental Science \& Policy. 54: 438-447. DOI: 10.1016/j.envsci.2015.08.012\\
\hline
\cellcolor{gray!10}{(Round 1978)} & \cellcolor{gray!10}{Round, P. D. (1978) An ornithological survey of the Somerset Levels 1976-77. A joint Wessex Water Authority and Royal Society for the Protection of Birds project.}\\
\hline
(Rowland et al. 2020) & Rowland, C.S.; Marston, C.G.; Morton, R.D.; O’Neil, A.W. (2020). Land Cover Map 1990 (25m raster, GB) v2. NERC Environmental Information Data Centre. https://doi.org/10.5285/1be1912a-916e-42c0-98cc-16460fac00e8\\
\hline
\cellcolor{gray!10}{(RSPB 2018)} & \cellcolor{gray!10}{RSPB. (2018). Sustainable Shores habitat creation opportunities WFS. Date Accessed: 01-07-2024. URL: https://opendata-rspb.opendata.arcgis.com/maps/24944d24920a445cb82b724c69715b59/about}\\
\hline
(Rural Payments Agency 2024) & Rural Payments Agency. (2024). RPA Parcel Points (England). Date Accessed: 29-08-2024. URL: https://environment.data.gov.uk/dataset/93a2433d-054a-484c-aec4-4e2ce9d7f2a9; Anonymised customer ID available on request to RPA (open.data@rpa.gov.uk).\\
\hline
\cellcolor{gray!10}{(Somerset Drainage Board Consortium 2011)} & \cellcolor{gray!10}{Somerset Drainage Board Consortium. (2011). Curry Moor Water Level Management Plan. Date Accessed: 29-08-2024. URL: https://somersetdrainageboards.gov.uk/environment/wlmps/}\\
\hline
(UKCEH 2021) & Based upon UKCEH Land Cover® Plus: Crops © 2021. UKCEH. © RSAC. © Crown Copyright 2007, Licence number 100017572\\
\hline

\end{longtable}

\endgroup{}




\end{document}
